\input{beamer_annales.sty}



%\usepackage[utf8]{inputenc}
\mode<presentation>
{
	\usetheme{Warsaw}
	\setbeamercovered{transparent}
}

\usepackage{times}

\usepackage{graphicx}
\usepackage{capt-of}
\usepackage{booktabs}
\usepackage{varwidth}

\newcommand\Fontvi{\fontsize{8}{8.2}\selectfont}

\usepackage{appendixnumberbeamer}

\usepackage[]{algorithm2e}

\usepackage{booktabs}

\usepackage[scale=2]{ccicons}

\usepackage{pgfplots}
\usepgfplotslibrary{dateplot}

\usepackage{mathtools}
\DeclarePairedDelimiter{\ceil}{\lceil}{\rceil}

\usepackage{xspace}
\newcommand{\themename}{\textbf{\textsc{metropolis}}\xspace}

\title[Asymptotics on the Lempel-Ziv 78 compression of Markov sources]
{Asymptotics on the Lempel-Ziv 78 compression of Markov sources}

\subtitle
{Exploring analytic information theory : from Markov source 
sampling to combinatorial analysis proofs}

\author[Duboc]
{Guillaume~Duboc}

\institute[
	Universities of Somewhere and Elsewhere]
{
	Computer Science Department\\
	Ecole Normale Sup�rieure de Lyon}

\date[M1 2018]
{M1 Internship, 2018}

\subject{Theoretical Computer Science}

\pgfdeclareimage[height=0.5cm]{logoENS}{logoENS}
\logo{\pgfuseimage{logoENS}}

% \date{\today}
% \titlegraphic{\hfill\includegraphics[height=1.5cm]{logo.pdf}}
\AtBeginSubsection[]
{
	\begin{frame}<beamer>{Table des mati�res}
	\tableofcontents[currentsection,currentsubsection]
\end{frame}
}

\expandafter\def\expandafter\insertshorttitle\expandafter{%
\insertshorttitle\hfill%
\insertframenumber\,/\,\inserttotalframenumber}

\begin{document}

\maketitle

\begin{frame}{Table of contents}
  \tableofcontents
\end{frame}


\section{The data compression problem}
\subsection{ Introduction to information sources }

\begin{frame}{ Words or sequences, and memoryless sources }

	\begin{block}{Definition: word or sequence or string}
		Given an alphabet $\mathcal{A}$, a \emph{\bfseries word} or \emph{\bfseries sequence}
		or \emph{\bfseries string} is an infinite sequence of random variables 
		$ X = {(X_k)}_{k \Nstar} $, each $X_k$ representing a symbol in $\mathcal{A}$.
	\end{block}

	\begin{block}{Definition : Bernoulli or Memoryless source}
		A source of information is a \emph{\bfseries Bernoulli} or \emph{\bfseries memoryless
		source} when all the symbols of $\mathcal{A}$ occur independently with a fixed probability.
		The word can be seen as an \emph{infinite sequence of Bernoulli trials}.
	\end{block}

\end{frame}

\begin{frame}{ Markov sources definition }

	\begin{block}{Definition: Markov source}
		An information source is a \emph{\bfseries Markov source} when 
		there is a \emph{\bfseries Markov dependency} between the consecutive symbols of a string.
	\end{block}

	\begin{block}{Definition: order of a Markov source}
		Let $V = |\mathcal{A}|$. A \emph{\bfseries Markov source} is of \emph{\bfseries order $r$}
		when the dependency can be encoded in a transition matrix of size $V^r \times V$, with 
		coefficients :
		\centers{$P(c | w)  \qquad  \forall\, (w, c) \in \mathcal{A}^r \times \mathcal{A}$}

		Informally : \emph{the probability that a symbols occurs depends on the previous $r$ symbols.}
	\end{block}

\end{frame}

\subsection{ The LZ78 compression scheme }

\begin{frame}{Description of the LZ78 algorithm}

	\begin{block}{ Algorithm }
		Given a word $w$.
		\begin{itemize}
			\item Initialize an empty dictionary
			\item While it is possible :
				\begin{description}
					\item Find longest prefix of $w$ that is not in the dictionary
					\item Add it to the dictionary, cut it from $w$
				\end{description}
		\end{itemize}
	\end{block}
\end{frame}

\begin{frame}
	\begin{block}{ Elements description }
		The data representation is $(\text{dictionary\_reference}, \text{symbol})$.
	\end{block}

	\begin{block}{ Remarks }
		The LZ78 algorithm builds a prefix tree from which the original word
		can be reconstructed.
	\end{block}
\end{frame}

\begin{frame}

	\begin{block}{ Definition : number of phrases }
		After compressing a word $w$, the number of phrases in the dictinary
		is noted $M(w)$.

		For words of size $n$, we write $M_n(w)$.
	\end{block}

	\begin{block}{ Code length }
		\centers{$ C(w) = \Sum{k=0}{M(w)} 
							\left( \ceil{\log_2(k)} 
								+ \ceil{\log_2(\mathcal{A})} \right) $}
	\end{block}

\end{frame}

\subsection{ The compression ratio, and entropy }


\begin{frame}

	\begin{block}{Definition : compression ratio}
		Let $w$ a word, and $C(w)$ its \emph{encoding} by a compression 
		algorithm.  The \emph{\bfseries	compression ratio} of $w$ is 
		$\f{|C(w)|}{|w|}$.
	\end{block}

	\begin{block}{ Main goals of compression algorithms }
		\begin{itemize}
			\item Improving the compression ratio
			\item Fast compression/decompression speed in Mb/s
		\end{itemize}
	\end{block}

	\begin{block}
		The tradeoff between these two goals is a sensitive research problem.
		Different compression standards :
		\begin{itemize}
			\item Google (Brotli, 2015)
			\item Facebook (Zstandard, 2016)
			\item Dropbox (DivANS, 2018)
		\end{itemize}
	\end{block}

\end{frame}


\begin{frame}{ Optimal encoding }
	\begin{block}{ Entropy of a Markov source }
		Let $\pi$ be a stationary distribution. 
		The entropy of a Markov chain is 
		\centers{$ h = - \Sum{i=1}{V} \pi \Sum{j=1}{V} p_{i j} \log(p_{i j}) $}
	\end{block}

	\begin{block}{ Optimality of LZ78 }
		Considering words of length $n$.

		\centers{$ \f{|C(w)|}{|w|} -h$ goes to zero for $n \rightarrow \pinf$}
	\end{block}
\end{frame}

\subsection{ Algorithmic improvements }
	\begin{frame}{ Optimal parsing}
	\end{frame}
	\begin{frame}{ Flexible parsing }
	\end{frame}

\section{Process evaluation}
\subsection{Theoretical models}


{\setbeamercolor{background canvas}{bg=white}
\begin{frame}{Markov Independent Model}
\begin{egalites}
  & X(1) 
    & {\color{red}{0}} {\color{green}{0}} 00000\dots \\
  &X(2) 
    & {\color{red}{1}} {\color{green}{0}}10101\dots \\
  & X(3) 
    & {\color{red}{10}} {\color{green}{0}} 1101 \dots \\
  & X(4) 
    & {\color{red}{00}} {\color{green}{1}} 100111\dots
\end{egalites}

\end{frame}}

\subsection{Experimental conditions}

\begin{frame}
	\begin{block}{ Coding details }
	\begin{itemize}
		\item Python code $\sim$ 2000 lines 
		\item Markov source sampling 
		\item Optimized datastructure (digital search tree)
		\item Parallelization
		\item Reproducibility of datasets
	\end{itemize}
	\end{block}

\end{frame}


\subsection{ Extracting results }

\begin{frame}{Central Limit Theorem confirmation}
	\includegraphics[width = \textwidth]{empirical_normalization_10e6_500.png}
\end{frame}

\begin{frame}{Hypothesis testing for the variance}

		\begin{block}{ Complex matrix }
			Defining $P(s)$ as 
			$
			\begin{array}{rl}
				p_{1 1}^{-s} & p_{1 2}^{-s} \\
				p_{2 1}^{-s} & p_{2 2}^{-s} 
			\end{array}
			$
		\end{block}

		\begin{block}{ Variance expression }
			\centers{$ V_n =  \left( \ddot{\lambda}(-1) - { \dot{\lambda}(-1) }^2 \right) \f{n}{\ln^2 n} $}
		\end{block}
\end{frame}

\begin{frame}
	\includegraphics[width = \textwidth]{eig_fig1.png}
\end{frame}


\section{ Analytic information theory }

\subsection{ Power series }
	\begin{frame}{ Definition, usage }
		\begin{block}{Definition}
			\centers{$ A(z) = \Sum{n\geq0}a_n z^n$}
		\end{block}

		\begin{block}{Remarks}
			\begin{itemize}
				\item Used as an algebraic item with the convolution product
				\item No convergence problems
			\end{itemize}
		\end{block}

	\end{frame}

\subsection{ Complex analysis tools }

	\begin{frame}

	\begin{block}{Poissonization and Depoissonization}
		\centers{${\tilde G}(z) = \Sum{n\geq 0}{} a_n \f{z^n}{n!} \ex{-z} $}
	\end{block}

	\begin{block}{Mellin transform}
		Make recurrence relation between random variables
		become linear in order to solve them more easily.
	\end{block}

	\end{frame}

\section{ Application to covariance analysis }

	\subsection{ Tail symbols }

		\begin{frame}{Tail symbols}

			\begin{block}{ Illustration }
				\begin{egalites}
	& X(1) 
		& {\color{red}{0}} {\color{green}{0}} 00000\dots \\
	&X(2) 
		& {\color{red}{1}} {\color{green}{0}}10101\dots \\
	& X(3) 
		& {\color{red}{10}} {\color{green}{0}} 1101 \dots \\
	& X(4) 
		& {\color{red}{00}} {\color{green}{1}} 100111\dots
	\end{egalites}
			\end{block}

			\begin{block}{Definition}
				Let $c$ be a character from our alphabet $\{ a, b \}$.
				In the case when all the sequences start with a $c$, we 
				define $T_n^{\,c}$ the \emph{number of times $a$ is a tail symbol in 
				the experiment}.
			\end{block}

		\end{frame}

		\begin{frame}{Definition and relation}

			

			\begin{block}{Recurrence}
				For $n \geq 0$, we have :
					\[ \boxed{ T_{n+1}^{\,c} = \delta_a + 
											{{\tilde T}_{N_a}}^a
											+ {{\tilde T}_{N_b}}^b } \]
			\end{block}
		
			\begin{block}{ Notations }
				\begin{itemize}
				\item $\delta_a = 
							\begin{cases} 
								1 & \text{if $a$ is the tail symbol of the
										first sequence}\\
								0 & \text{else} 
							\end{cases}$

				\item $N_a$ is the 
				random variable giving \emph{the size of the left subtree 
				which contains phrases whose second letter is $a$}

				\item ${{\tilde T}_{N_a}}^a$ is the number of 
				times $a$ is a tail symbol for the sequences that were 
				used to  build the subtree with phrases having $a$ as second
				symbol.

				\item $T_0^c$ for all $c$ by convention.
			\end{itemize}
			\end{block}

		\end{frame}

		\begin{frame}{Total path lenght}

			\begin{block}{Definition}
				Defining $L_n^{\,c}$ as the \emph{total 
				path length of the nodes of the DST that was built with
				MI model with $n$ sequences starting with letter $c$}.
				It is the sum of the lengths of all the prefix phrases.
			\end{block}

			\begin{block}{Recurrence relation}
				For all $n\geq 0$ :

				\[
				\boxed{ 
					L_{n+1}^c = n + 
										{{\tilde L}_{N_a}}^a + 
										{{\tilde L}_{N_b}}^b
				}
				\]
			\end{block}
		\end{frame}

	\subsection{Simulation results}

		\begin{frame}
			Inconclusive, but informative
		\end{frame}

	\subsection{Analytic solution}

		\begin{frame}
			\begin{block}{Recurrence}
			  \[ \boxed{ \Cov(T_{n+1}^{\,c}, L_{n+1}^c) = 
          \Cov ( {{\tilde T}_{N_a}}^a,
                         {{\tilde L}_{N_a}}^a )
          + \Cov ( {{\tilde T}_{N_b}}^b, 
                          {{\tilde L}_{N_b}}^b ) } \]
			\end{block}
		\end{frame}

		\begin{frame}
			\begin{block}{Poisson transform}
				Defining 
  \[ \boxed{ C_c(z) 
            = \Sum{n\geq0}{} \Cov(T_n^{\,c}, L_n^{\,c}) 
                            \f{z^n}{n!} \ex{-z} } \]
			\end{block}

			\begin{block}{Differential equation}
							\[
			\boxed{
			\partial_z C_c(z) + C_c(z) 
				= C_a(zp) + C_b(zq)
			}
			\]
			\end{block}
		\end{frame}

\begin{frame}
	\includegraphics[width=\textwidth]{eig_fig2.png}
\end{frame}


\begin{frame}
	\includegraphics[width=\textwidth]{eig_fig3.png}
\end{frame}


\begin{frame}
	\includegraphics[width=\textwidth]{eig_fig4.png}
\end{frame}


\begin{frame}
	\includegraphics[width=\textwidth]{eig_fig5.png}
\end{frame}

\begin{frame}[allowframebreaks]
\nocite{*}
\bibliographystyle{unsrt}
\bibliography{sample}
\end{frame}

\end{document}
