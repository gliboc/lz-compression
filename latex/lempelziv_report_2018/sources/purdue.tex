During my internship, I worked in the Felix Haas building 
which notably hosts the Center for Science of Information - a research
lab dedicated to information theory problematics. 
Early after my arrival, I started collaborating with some of 
my supervisor's contacts overseas: namely Philippe Jacquet in Paris 
Nokia Labs, and an Italian research team in Palermo -
but I also had the pleasure to discuss, work and spend some time with 
several postdoctoral researchers at Purdue, 
working on Graph Dependency Theory (Abram 
Magner), Information Theory of Communication (Arun Padakandla) and 
other interesting subjects. 
Thanks to them I acquired precious insights 
into the academic environment in the United States. Speaking of which : 
I am sincerely and very thankful to the CSoI and its administration 
for all their help in hosting and organizing my trip. Any trouble 
I could have had was considered and I was able to focus on research 
and writing in very good and peaceful conditions.
Overall, working at Purdue
was a great opportunity which I would recommend to anyone wanting to expand 
their mind about academia and its power to connect people overseas.
It has stirred my interest about doing research abroad - in particular 
in the U.S., and I look forward for another occasion to do so.