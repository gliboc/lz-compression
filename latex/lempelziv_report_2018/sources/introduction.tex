\section{ Introduction }

\emph{Data compression}, also called \emph{source coding}, or \emph{bit-rate reduction} involves encoding information using fewer bits than the original representation.

\subsection{ Introduction to the problem }

Data compression is achieved by different types of algorithms, and it can either
be lossy or lossless. In order to study it precisely, we define a data compression
scheme. First, we define an algorithm which operates on words. Then, we introduce
a probabilitic model for the data to be compressed, which allows us to quantify
the efficiency of our algorithm.

\subsubsection{ Probabilistic models }

\begin{df}
    \label{def:source}
    Let $\mathcal{A}$ be an alphabet.
    An \emph{information source} is a one-sided infinite sequence of random
    variables $(X_k)_{k=1}^{\pinf}$ with each $X_k$ 
    having values in $\mathcal{A}$.
\end{df}

\begin{rmk}
    \label{rmk:sequence}
    Each realization of an information source is called a 
    \emph{sequence} or \emph{word}.
\end{rmk}

\begin{rmk}
    \label{rmk:source}
    \emph{Defining} the law of the $X_k$ produces out different
    models for data generation which can be studied mathematically
    and \emph{simulated}.
\end{rmk}

\begin{df}
    \label{def:memoryless}
    A \emph{memoryless source} is an \emph{information source}
    for which the $X_k$ are mutually independent, following
    the uniform law on $\mathcal{A} = \{ a_1, \dots, a_V \}$ :
    \centers{$
        \proba{} \left( X_k = a_k \right) = p_k 
        \qquad \textmd{with } \,\Sum{i=1}{V} \,p_i = 1
    $}
\end{df}

\begin{rmk}
    \label{rmk:memoryless}
    This is the simplest information source and it has been 
    studied successfully in the past, but it is not a realistic 
    model. We replace it with the following Markov model whenever 
    possible.
\end{rmk}

\begin{df}
    \label{def:markov}
    A \emph{Markov source} is an \emph{information source}
    with a Markov dependency between successive symbols.
\end{df}

\begin{df}
    \label{def:markovorder}
    A \emph{Markov source of order $r$} is a Markov source
    for which each symbol apparition depends on the previous 
    $r$ symbols.
\end{df}

\begin{rmk}
    \label{rmk:markov2}
    We will study Markov sources of order 1 - where each
    symbol simply depends on the previous one. This is 
    general enough, as Markov sources of superior order
    can be simulated by expanding the alphabet and 
    using a Markov source of order 1.
\end{rmk}


\subsubsection{ Lempel-Ziv 78 }

In general, Lempel-Ziv algorithms exploit previously seen 
redundancy to save off coding space. The Lempel-Ziv 78 does
so by constructing a prefix tree which allows to describe 
parts of a word by referring to previously seen phrases.

%TODO write algorithm



\subsubsection{ Probabilistic analysis }

Under this context, we can define and conduct a thorough
analysis of several random variables with different meanings
regarding the effectiveness of compression.

\begin{nota}
    \label{nota:universe}
    Let $n$ be an integer - the size of the considered words.
    Defining $\Omega_n$ the set of words of size $n$ on alphabet
    $\mathcal{A}$. Each word being an event, a natural probability
    space is given by considering the output of size $n$ of a Markov
    source.
\end{nota}

\begin{rmk}
    \label{rmk:probaspace}
    We will now study random variables defined 
    on this probability space. 
\end{rmk}

\begin{nota}
    \label{nota:output}
    The output of a Markov process is denoted by 
    the random variable $W$.
\end{nota}

\begin{nota}
    \label{nota:numberphrases}
    The \emph{number of phrases} used to compress a word $W$ with
    LZ78 is given by $M_n(W)$ or simply $M_n$.
\end{nota}

\begin{rmk}
    \label{rmk:numberphrases}
    This is one of the most important variables to consider because,
    as it will appear shortly, it is closely tied to the \emph{compression 
    ratio} of LZ78.
\end{rmk}

\begin{df}
    \label{df:codelength}
    The \emph{codelength} is the \emph{number of bits} required 
    to encode the LZ78-compressed version of a word, denoted by $C(W)$.
\end{df}

\begin{df}
    \label{df:compratio}
    The \emph{compression ratio} is the ratio between the 
    codelength and initial size of a word, $\f{C(W)}{|W|}$.
\end{df}

