\begin{tabular}{ll}
    &\textbf{Link:} \url{https://github.com/gliboc/lz-compression}\\
    &{\color{gray} 2500} lines of code in \emph{Python}
\end{tabular}

\bigskip
\noindent
This code repository is able to simulate two models 
from~\cite{jacquet_average_2001} : the \emph{\bfseries Lempel-Ziv model}, 
where a sequence of $n$ symbols is generated by an information
source and compressed by a LZ'78 algorithm, as well as the 
\emph{\bfseries Markov-Independent model}, where $n$ sequences are generated 
and then sequentially given to LZ'78 to generate phrases,
the $i$th phrase being the shortest prefix of the $i$th
sequence that was not seen before as a phrase. 

\bigskip
\begin{itemize}[nosep]
    \item[] {\color{gray} markov.py } 
    \item[] \quad Markov source implementation 
    \item[] {\color{gray} parallel\_tails.py} 
    \item[] \quad Fast and parallelized generation of sequence
    for the MI model
    \item[] {\color{gray} lempelziv.py} 
    \item[] \quad Lempel-Ziv'78 algorithm implementation
    \item[] {\color{gray} neininger.py  szpan.py  eigenvalues.py  normal.py} 
    \item[] \quad Functions for computing theoretical 
    expressions of mean and variance
    \item[] {\color{gray} main.py  tails.py} 
    \item[] \quad Plots graphs for visualization
    \item[] {\color{gray} experiments.py  experiment\_tails.py} 
    \item[] \quad Managing data generated for experiments
\end{itemize}

\bigskip
This repository also contains a variety of \emph{Jupyter} notebooks,
which might provide some insight into my work method. This routine 
consisted in writing code to run numerics intertwined with exact 
expression and computations written in \LaTeX. I found it to be
quite productive in terms of getting results and writing and presenting 
them. The notebooks are not polished work but drafts.