\hypertarget{definitions}{\section{Definitions}}

\noindent
These are definitions and notations in order
to write the proof of~\cite{langiu_speed_2018}[Theorem 1] - which
is an asymptotic upperbound used to prove a quantity goes to 
0 exponentially as $n$ goes to $\pinf$. I wrote those as part 
of my report on the paper, in order to make cleared the use 
of random variables throughout their proof, and lately added some 
more context about the flexible parsing algorithm.

% part  (end)

\begin{df}
For all $n\in\mathbb{N}$, calling $\Omega_n$ the set 
of words of length $n$.
\end{df}

\begin{df}
Defining $W \in \Omega_n$ to be a random variable which outputs 
words of length $n$ from a memoryless source.
\end{df}

\begin{df}
Considering $J \in \mathbb{N}^{\star}$ to be a random
variable which, in the event $\{ W = w \}$,
uniformly randomly picks the index of
one of the phrases of $w$.
The joint law of $J$ with $W$ being :

\centers{$ \proba{ W=w, J=j } = 
                    \begin{cases}
                        \tf{1}{M_n(w)} \quad & \text{ if } j \leq M_n(w) \\
                        0 \qquad \qquad  &\text{ else }
                    \end{cases} $}
\end{df}

\begin{rmk}
We might choose another randomness for $J$, but this one seems
more natural.
\end{rmk}

\begin{df}
For a given word $w\in\Omega_n$, 
and for all $i\in\mathbb{N}^{\star}$, 
we consider $g_w(i)$ defined by
\centers{$g_w(i) = f_w(i) + |L_{f(i)}|$ }
where $f_w(i)$ is the starting index
of the $i^{\text{th}}$ phrase of the flexible parsing of $w$,
and $|L_{f_w(i)}|$ is the length of the longest greedy phrase
given by the Lempel-Ziv parsing of this same word.
\end{df}

\begin{df}
We define $g_{\scriptscriptstyle W}(J)$ to be the random variable which
outputs $g_w(j)$ during the events $\{ W=w \}$ and $\{ J = j \}$.
\end{df}

\begin{df}
For all $k \in \mathbb{N}$, let $L_{g_{\scriptscriptstyle W}(J) - k}$ be the random
variables which gives the $(k+1)^{\text{th}}$ possible phrase 
for the flexible parsing at index $g_{\scriptscriptstyle W}(J) - k$. 
Its only randomness comes from $W$ and $J$. 
If $i\leq 0$, we might assume that $L_i$ will be the empty word of 
size 0.
\end{df}


\begin{nota}
    Denoting by
    \centers{$B_{J, W}^k = | L_{g_{\scriptscriptstyle W}(J) - k} |$}
    \noindent the length of this $(k+1)^{\text{th}}$ candidate.
\end{nota}


\noindent
We can now study the random variable
    \centers{$ \underset{ 0 \leq k \leq g_{\scriptscriptstyle W}(J) }{ \max } 
                    \left\{ { B_{J, W}^k - k } \right\} $}

\noindent
Given any $(j,w) \in \mathbb{N}^{\star} \times \Omega_n$,
under the events $\{ J = j \}$, $\{ W = w \}$ and $\{ J \leq M_n(W) \}$, 
this random variable is the length of 
the $j^{\textmd{th}}$ phrase of the flexible parsing of the word $w$.

\begin{df}
Defining the random variable $D_n^{FLEX}$ as 

\begin{egalites}
    &D_n^{FLEX} (w) 
        & \f{1}{M_n(w)} \Sum{j=0}{M_n(w)-1} |u_j^{FLEX}(w)| \\
        && \f{1}{M_n(w)} \Sum{j=0}{M_n(w)-1}  
                                \underset{ 0 \leq k \leq g_w(j) }{ \max } 
                                \left\{ { B_{j, w}^k - k } \right\}
\end{egalites}
\noindent
where $D_n^{FLEX}(w)$ is the empirical average of the lengths of the phrases
of the flexible parsing of a word $w$, contrary to
        $\underset{ 0 \leq k \leq g_w(J) }{ \max } 
        \left\{ { B_{J, w}^k - k } \right\} $,
with $J$ a random variable and $w$ fixed,
which is the length of a uniformly randomly selected 
phrase of the flexible parsing of $w$. 
\end{df}

\begin{df}
    We denote $x_n$ the average value of ${D_n}^{\text{LZ}}$ :
    \centers{$x_n = \f1{h} \log_2 \pa{ \f{nh}{\log_2(n)} }$}
\end{df}

\begin{rmk}
    The random variable ${D_n}^{\text{LZ}}$ is the
    length of a phrase randomly taken from the 
    Lempel-Ziv parsing 
    of a memoryless-generated word. It is not the same as
    the empirical average length of a phrase, which we can
    denote,
    for $w \in \Omega_n$ :
    \centers{$ \overline{D_n}^{\text{LZ}}(w)
                = \f{1}{M_n(w)}
                    \SUM{j=0}{M_n(w)-1} \left| 
                                        u_j^{\text{FLEX}}
                                        \right| $}
\end{rmk}

\begin{nota}
    We denote ${b_n}^{\delta}$ as
    \centers{${b_n}^{\delta} = x_n + (c_3 x_n)^{\delta}$}
\end{nota}

\noindent
We will study
    \centers
    {$ \proba{ 
        \underset{ 0 \leq k \leq g_{\scriptscriptstyle W}(J) }{ \max } 
        \left\{ { B_{J, W}^k - k } \right\}
            > {b_n}^{\delta} } $}