\section{Optimal parsing}

Among my other research work at Purdue, I was asked to review the draft
of a paper~\cite{langiu_speed_2018} by Italian researchers which have collaborated 
with my supervisor in the past. Their current field of study is Optimal Parsing in the case of 
LZ'77 and LZ'78 - which can be resumed as a dynamic partionning of a data 
sequence into blocks in order to improve the efficiency of the LZ algorithms.
They explore the question which arised from practical use~\cite{mahoney_large_nodate}:
why is LZ'77 in practice more efficient than LZ'78, even though in theory LZ'78 is 
asymptotically better than LZ'77 as its compression ratio converges towards
the entropy lower bound faster than LZ'77 ?
This was an interesting work
as it gave me the occasion to learn, understand and apply the Flexible Parsing 
algorithm, and make some contributions to its analysis.
I wrote a report which we sent to them over the course of June, and I will now present
excerpts from it, among with some of the context around their work.

% \subsection{LZ77 and LZ78 comparison }

% \subsubsection{Practical uses of LZ77 }

% Although it is based on the same approach of identifying 
% previously seen phrases - but storing them in a fixed size 
% dictionary instead of an evergrowing parse tree like in LZ78 -
% the LZ77 algorithm is the most used in practical implementations
% of universal compression algorithms.

% Let us dive in on a few of the most recent ones :

% \begin{itemize}
%     \item DivANS
%     \item zStandard
%     \item 
% \end{itemize}


\renewcommand{\subsection}{\subsubsection}
\renewcommand{\section}{\oldsubsection}

\hypertarget{definitions}{\section{Definitions}}

\noindent
Definitions and notations I wrote in order
to write the proof of~\cite{langiu_speed_2018}'s first theorem - which
is an upperbound used to prove a quantity goes to 
0 exponentially as $n$ goes to $\pinf$ - can be found in Appendix~\ref{app:definitions}. 
I wrote those as part of my report on the paper, in order to make clearer the use 
of random variables through it. This report, which can be found entirely
in the Appendix~\ref{app:flexible}, is not easy to understand without the context and 
notations of the draft. This is why in the following section I just describe my contributions
while linking to the corresponding parts of the report, not trying to explain the subject 
entirely.


\section{Remarks}
First, the summary of remarks and suggestions we came up with
my supervisor on their paper can be found in Appendix~\ref{app:corrections}.

\hypertarget{computations}{\section{Computations}}
To prove that the definitions I introduced were useful to reason 
and write proofs on the subject, I rewrote some computations which
were, in their actual form in the draft, not formally accurate though
understandable. See Appendix~\ref{app:computations}.


\section{Proof suggestions}
\noindent
\hypertarget{critics}{}

While working with my supervisor on this paper, we felt that 
an approach to solving their problem was to use conditionning
in their proof. In a nutshell, their argumentary around 
algorithm was correct \emph{on average} - which means almost 
surely in the sense of probabilities 
- and conditionning could
prove a good method to quantify the uncertainty around their
analysis in a formal proof. The suggestions I wrote 
in order to begin to formalize and explore this track of proof 
are in Appendix~\ref{app:critics}.


\section{Proof of an upperbound for the paper}
\hypertarget{upperbound}{}

Finally, while working at the white board with Wojciech, he suggested 
I try a simpler proof for establishing one of the upperbounds in the 
paper goes to 0 as $n$ goes to $\pinf$. This method was, happily, 
a success - and I hope it helps their analysis going forward.
The full technical proof can be found in Appendix~\ref{app:upperbound}.

\section{Conclusion}

I found that my work on this paper draft was a good exercise in reading
and technical writing. It also broadened my views over data compression, 
even though it was not directly related to my main research tasks, which were
centered on LZ'78. I am looking forward to continuing this analysis 
with the italian research team.

\renewcommand{\subsection}{\oldsubsection}
\renewcommand{\section}{\oldsection}