\section{ Optimal parsing}

Among my other research work at Purdue, I was asked to review the draft
of a paper~\ref{langiu_speed_2018} by Italian researchers which have collaborated 
with my supervisor in the past. Their current field of study is Optimal Parsing in the case of 
LZ'77 and LZ'78 - which can be resumed as a dynamic partionning of a data 
sequence into blocks in order to improve the efficiency of the LZ algorithms.
They explore the question which arised from practical use~\ref{mahoney_large_nodate}:
why is LZ'77 in practice more efficient than LZ'78, even though in theory LZ'78 is 
asymptotically better than LZ'77 as its compression ratio converges towards
the entropy lower bound faster than LZ'77 ? This was an interesting work
as it gave me the occasion to understand and apply the Flexible Parsing 
algorithm, and make some minor contributions to its analysis and presentation.
I wrote a report which I sent them over the course of June and I will now present
excerpts from it, among with some of the context around their work.

% \subsection{ LZ77 and LZ78 comparison }

% \subsubsection{ Practical uses of LZ77 }

% Although it is based on the same approach of identifying 
% previously seen phrases - but storing them in a fixed size 
% dictionary instead of an evergrowing parse tree like in LZ78 -
% the LZ77 algorithm is the most used in practical implementations
% of universal compression algorithms.

% Let us dive in on a few of the most recent ones :

% \begin{itemize}
%     \item DivANS
%     \item zStandard
%     \item 
% \end{itemize}


\renewcommand{\subsection}{\subsubsection}
\renewcommand{\section}{\oldsubsection}

\hypertarget{definitions}{\section{Definitions}}

\noindent
These are definitions and notations in order
to write the proof of~\cite{langiu_speed_2018}[Theorem 1] - which
is an asymptotic upperbound used to prove a quantity goes to 
0 exponentially as $n$ goes to $\pinf$. I wrote those as part 
of my report on the paper, in order to make cleared the use 
of random variables throughout their proof, and lately added some 
more context about the flexible parsing algorithm.

% part  (end)

\begin{df}
For all $n\in\mathbb{N}$, calling $\Omega_n$ the set 
of words of length $n$.
\end{df}

\begin{df}
Defining $W \in \Omega_n$ to be a random variable which outputs 
words of length $n$ from a memoryless source.
\end{df}

\begin{df}
Considering $J \in \mathbb{N}^{\star}$ to be a random
variable which, in the event $\{ W = w \}$,
uniformly randomly picks the index of
one of the phrases of $w$.
The joint law of $J$ with $W$ being :

\centers{$ \proba{ W=w, J=j } = 
                    \begin{cases}
                        \tf{1}{M_n(w)} \quad & \text{ if } j \leq M_n(w) \\
                        0 \qquad \qquad  &\text{ else }
                    \end{cases} $}
\end{df}

\begin{rmk}
We might choose another randomness for $J$, but this one seems
more natural.
\end{rmk}

\begin{df}
For a given word $w\in\Omega_n$, 
and for all $i\in\mathbb{N}^{\star}$, 
we consider $g_w(i)$ defined by
\centers{$g_w(i) = f_w(i) + |L_{f(i)}|$ }
where $f_w(i)$ is the starting index
of the $i^{\text{th}}$ phrase of the flexible parsing of $w$,
and $|L_{f_w(i)}|$ is the length of the longest greedy phrase
given by the Lempel-Ziv parsing of this same word.
\end{df}

\begin{df}
We define $g_{\scriptscriptstyle W}(J)$ to be the random variable which
outputs $g_w(j)$ during the events $\{ W=w \}$ and $\{ J = j \}$.
\end{df}

\begin{df}
For all $k \in \mathbb{N}$, let $L_{g_{\scriptscriptstyle W}(J) - k}$ be the random
variables which gives the $(k+1)^{\text{th}}$ possible phrase 
for the flexible parsing at index $g_{\scriptscriptstyle W}(J) - k$. 
Its only randomness comes from $W$ and $J$. 
If $i\leq 0$, we might assume that $L_i$ will be the empty word of 
size 0.
\end{df}


\begin{nota}
    Denoting by
    \centers{$B_{J, W}^k = | L_{g_{\scriptscriptstyle W}(J) - k} |$}
    \noindent the length of this $(k+1)^{\text{th}}$ candidate.
\end{nota}


\noindent
We can now study the random variable
    \centers{$ \underset{ 0 \leq k \leq g_{\scriptscriptstyle W}(J) }{ \max } 
                    \left\{ { B_{J, W}^k - k } \right\} $}

\noindent
Given any $(j,w) \in \mathbb{N}^{\star} \times \Omega_n$,
under the events $\{ J = j \}$, $\{ W = w \}$ and $\{ J \leq M_n(W) \}$, 
this random variable is the length of 
the $j^{\textmd{th}}$ phrase of the flexible parsing of the word $w$.

\begin{df}
Defining the random variable $D_n^{FLEX}$ as 

\begin{egalites}
    &D_n^{FLEX} (w) 
        & \f{1}{M_n(w)} \Sum{j=0}{M_n(w)-1} |u_j^{FLEX}(w)| \\
        && \f{1}{M_n(w)} \Sum{j=0}{M_n(w)-1}  
                                \underset{ 0 \leq k \leq g_w(j) }{ \max } 
                                \left\{ { B_{j, w}^k - k } \right\}
\end{egalites}
\noindent
where $D_n^{FLEX}(w)$ is the empirical average of the lengths of the phrases
of the flexible parsing of a word $w$, contrary to
        $\underset{ 0 \leq k \leq g_w(J) }{ \max } 
        \left\{ { B_{J, w}^k - k } \right\} $,
with $J$ a random variable and $w$ fixed,
which is the length of a uniformly randomly selected 
phrase of the flexible parsing of $w$. 
\end{df}

\begin{df}
    We denote $x_n$ the average value of ${D_n}^{\text{LZ}}$ :
    \centers{$x_n = \f1{h} \log_2 \pa{ \f{nh}{\log_2(n)} }$}
\end{df}

\begin{rmk}
    The random variable ${D_n}^{\text{LZ}}$ is the
    length of a phrase randomly taken from the 
    Lempel-Ziv parsing 
    of a memoryless-generated word. It is not the same as
    the empirical average length of a phrase, which we can
    denote,
    for $w \in \Omega_n$ :
    \centers{$ \overline{D_n}^{\text{LZ}}(w)
                = \f{1}{M_n(w)}
                    \SUM{j=0}{M_n(w)-1} \left| 
                                        u_j^{\text{FLEX}}
                                        \right| $}
\end{rmk}

\begin{nota}
    We denote ${b_n}^{\delta}$ as
    \centers{${b_n}^{\delta} = x_n + (c_3 x_n)^{\delta}$}
\end{nota}

\noindent
We will study
    \centers
    {$ \proba{ 
        \underset{ 0 \leq k \leq g_{\scriptscriptstyle W}(J) }{ \max } 
        \left\{ { B_{J, W}^k - k } \right\}
            > {b_n}^{\delta} } $}

\question{Corrections}

\begin{itemize}

\item The formal definition of ${D_n}^{\text{LZ}}$ in $\numero{3}$ is 
      misleading because it contradicts the previous verbal definition.
      This formula is the empirical average length of a phrase, whereas
      ${D_n}^{\text{LZ}}$ is used in the rest of the paper as the length
      of a randomly selected phrase. These two aren't equal :
      if we build a Lempel-Ziv DST from a word, then 
      ${D_n}^{\text{LZ}}$ can be seen as the depth of a random node, 
      which is different from the average path length computed 
      on all the nodes. 

\item In \emph{Remark 2}, I believe the definition of $v$ and 
      $t$ should rather be 
      $ v = a_{i'} \dots a_{i} $ and
      $ t = a_{i+1} \dots a_n$. 

\item The result $\numero{14}$ should be an equality, and actually it is one
      because of the flexible parsing algorithm. A proof by contradiction
      can show this. However, we upper bound 'g(j)' by $\pinf$ in 
      $\numero{23}$ so I currently see no purpose to using $\numero{14}$ (as an 
      equality) instead of $\numero{13}$. Furthermore, an equality is 
      needed to establish result $\numero{22}$.

\item The proof and reasoning around \emph{Theorem 1} has several flaws.
      The notation $X$ for a sequence depending on $n$, which isn't a 
      random variable, is misleading. On the other hand, it should appear
      that both $g$ and $j$ are random variables, as the randomness of 
      $j$ is used in the end of the proof. As for the arguments that 
      link $|L_{g(j)-k}|$ to ${D_n}^{\text{LZ}}$, I have
      indicated how I think they should be developped in the next section.

\item \emph{Theorem 2} is false as stated: we proved \emph{Theorem 1}
      using a random $j$. The randomness remains, so the quantifier
      'for any $j < M_n$' should be removed. This would be true for 
      \emph{Theorem 1} too.

\item The proof of \emph{Theorem 2} may stop at $\numero{26}$
      since we can directly prove this upperbound goes to $0$.
      This yields a tighter upperbound for \emph{Theorem 2}.
      I detailed this analysis in the last section of this report.

\item In that same proof, the step between $\numero{25}$ and 
      $\numero{26}$ relies heavily on a result from $\pac{6}$.
      It seems not obvious that the result can be applied as such.

\end{itemize}

\subsection{Some probabilistic computations}
\label{app:computations}
With these definitions, 
we can do the formal computations at the beginning 
of the proof of \emph{Theorem 1}.

By conditionning on $W$ and $J$, we obtain

\begin{calculs}
        &\hspace{2cm} \proba{ \underset{ 0 \leq k \leq g_W(J) }{ \max } 
        \left\{ { B_{J, W}^k - k } \right\} > {b_n}^{\delta} } \\
            &&=& \Sum{w \in \Omega_n}{} \Sum{j \in \mathbb{N}^{\star}}{} \,
                \proba{ \underset{ 0 \leq k \leq g_W(J) }{ \max } 
        \left\{ { B_{J, W}^k - k } \right\} > {b_n}^{\delta}, W=w, J=j } \\[5mm]
            
            &&=& \Sum{w \in \Omega_n}{} \Sum{j \in \mathbb{N}^{\star}}{} \,
                \proba{ \union{k=0}{g_w(j)} \left\{ B_{w, j}^k > k + {b_n}^{\delta} \right\} 
                            ,W=w, J=j }\\[5mm]
            
            &&\leq&
                \Sum{w \in \Omega_n}{} \Sum{j \in \mathbb{N}^{\star}}{}
                \Sum{k=0}{g_w(j)} 
                \proba{ B_{w, j}^k > k + {b_n}^{\delta}, W=w, J=j } \\[5mm]
            
            &&\leq&
                \Sum{w \in \Omega_n}{} \Sum{j \in \mathbb{N}^{\star}}{}
                \Sum{k=0}{\pinf} \,
                \proba{ B_{w, j}^k > k + {b_n}^{\delta}, W=w, J=j }
                 \\[5mm]

            &&=&
                \Sum{k=0}{\pinf}
                \Sum{w \in \Omega_n}{} \Sum{j \in \mathbb{N}^{\star}}{} \,
                \proba{ B_{w, j}^k > k + {b_n}^{\delta}, W=w, J=j }
                 \\[5mm]

            &&=&
                \Sum{k=0}{\pinf} \,
                \proba{ B_{W, J}^k > k + {b_n}^{\delta} } \\
\end{calculs}




\noindent
For all $k\in\mathbb{N}$, we may now prove that
    \centers{
        $ \proba{ B_{J, W}^k > k + {b_n}^{\delta} } 
            \leq \proba{ {D_n}^{\text{LZ}} > k + {b_n}^{\delta} } $
    }












\subsection{Proof suggestions}
\noindent
Let $k \in\mathbb{N}$.
Currently, the proof to show this stands on three arguments :

\begin{itemize}
    \item[(1)] The first is that $L_{g_W(J)-k}$ can be seen as 
          a random phrase from the Lempel-Ziv parsing of 
          a word of length $N$, where $N \leq g_W(J)$.

    \item[(2)] The second, is that $|L_{g_W(J)-k}|$ can 
          therefore be considered the same as ${D_N}^{\text{LZ}}$
          \textit{i.e} at least equal in law.

    \item[(3)] The third is that, for all $n'\leq n$, 
        ${D_n'}^{\text{LZ}} \leq {D_n}^{\text{LZ}}$.
\end{itemize}

There are different problems with each of these arguments :

\begin{itemize}
    \item The randomness of $L_{g_W(J)-k}$ is not clearly
          established. For example, what is $N$ ? In order
          to establish this result, we have to define it
          to be able to even consider ${D_N}^{\text{LZ}}$.

    \item Even if we can identify $N$ and, let's say,
          condition it with $\{ N = n' \}$, it is not obvious
          that the choice of a phrase in ${g_W(J)-k}$
          is the same as the uniform choice that operates
          in ${D_n'}^{\text{LZ}}$.

    \item As for (3), this result is true on average, but 
          not always. Indeed, since ${D_n}^{\text{LZ}}$
          (resp. ${D_n'}^{\text{LZ}}$) is concentrated
          around $x_n$ (resp. $x_n'$), and $x_n'$ < $x_n$
          since $n' < n$, we can show 
          that this result holds with high probability.
\end{itemize}

\section{Proof of an upperbound for the paper}
\hypertarget{upperbound}{\centers{\question{Upperbound proof}}}
 
 This is a proof that the upperbound 
in $\numero{26}$ goes to 0. Assuming


\centersright{ $ \SUM{k=0}{\pinf} 
            \proba{ D_n^{LZ} (W) > k + b_n^{\,\delta} }
            \leq 
            A \alpha^{ {(c_3 x_n)}^{\delta - \tf12} } 
                \Sum{i=0}{\pinf} \alpha^{ \tf{ i }{ \sqrt{ c_3 x_n } } }
        $}{ \numero{26} }

\noindent We can prove directly that the upperbound term goes to 0 
for $n$ going to $\pinf$,
without resorting to another majoration.
Since $\sqrt{c_3 x_n}$ goes to infinity for $n \rightarrow \pinf$,
we can pick $n$ such that $\sqrt{c_3 x_n} > 1$. Therefore
$\alpha^{\tf{1}{\sqrt{c_3 x_n}}} < 1$ and the geometric sum gives

\centers{ $ \SUM {i=0}{\pinf}
            \alpha^{ \tf{i}{\sqrt{c_3 x_n}} } =
                  \f { 1 } 
                     { 1 - \alpha^{ \tf 1 { \sqrt{ c_3 x_n } } } } $}

\noindent It remains to prove that 
    \centers{$ \underset{n\rightarrow\pinf}{\limt} \,\,\,
                  \f { \alpha^{ {(c_3 x_n)}^{ \delta - \tf12} } } 
                     { 1 - \alpha^{ \tf 1 { \sqrt { c_3 x_n } } }} = 0
            $}

which is done by using L'Hospital's rule.
Define :
\centers
    {$ \fonction
            {f}
            { \Rplusstar }
            { \R }
            { x }
            { \alpha^{ x^{ \delta - \tf12 } } }
      \qquad \text{ and } \qquad
      \fonction
        {g}
        { \Rplusstar }
        { \R }
        { x }
        {  1 - \alpha^{ \f1{ \sqrt{x} } } } $}

\noindent Let $x \in \intoo{0}{\pinf}$. Derivating yields :

\centers{$ f'(x) = \ln \alpha \pa {\delta - \f12}
                    \, x^{\delta - \tf32}
                    \, f(x)
            \qquad \text{ and } \qquad
     g'(x) 
        =  \ln \alpha\,\, \f 1 {2x\sqrt{x}}\, \alpha ^ {\f 1 {\sqrt{x}}} $}

\hypertarget{upperbound}{there}
We proceed to show that $\f{f'(x)}{g'(x)}$ goes to 0 as $x$ goes
to $\pinf$. We have

   \centers {
        $ \f{f'(x)}{g'(x)}
             = \f { \pa{\delta - \f12}
                    \, x^{\delta - \tf32}
                    \alpha^{ x^{ \delta - \tf12 } } }
                  { \f{1}{2x\sqrt{x}} 
                        \alpha^{ \f{1}{\sqrt{x}} } }
        = 2 \pa{\delta -\f12} x^{\delta} 
            \cdot \f { \alpha^{x^{\delta-\tf12}}}
                      { \alpha^{ \f{1}{\sqrt{x}} } } $} 

Since $\alpha^{\f{1}{\sqrt{x}}} 
        \underset{x\rightarrow\pinf}{\longrightarrow} 1$, we are left to study 
        $ x^{\delta} \alpha^{x^{\delta-\tf12}}$.

 

\begin{egalites}
 Writing & x^{\delta} \alpha^{x^{\delta-\tf12}}
        & \ex{ \delta \log x + \log \alpha \cdot x^{\delta - \tf12}}
\end{egalites}

and taking the log, since $\delta > \tf12$ and $\log \alpha < 0$ we see that

\centers{$ \delta \log x + \log \alpha \cdot x^{\delta - \tf12} 
            \underset{x\rightarrow\pinf}{\longrightarrow} \minf$}

Therefore $x^{\delta} \alpha^{x^{\delta-\tf12}} \underset{x\rightarrow\pinf}{\rightarrow} 0$,
and given that $f(0) = g(0) = 0$, L'Hospital's rule applies, proving that 
$\f{f(x)}{g(x)} \underset{x\rightarrow\pinf}{\longrightarrow} 0$.






            



\renewcommand{\subsection}{\oldsubsection}
\renewcommand{\section}{\oldsection}