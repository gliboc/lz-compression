\titre{ Covariance asymptotics }
% Veuillez ne pas modifier ce titre SVP.
% En cas de doute, pr�venez votre coordinateur.

%%%%%%%%%%%%%%%%%%%%%%%%%%%%%%%%%%%%%
\section{Markov Independent Model}

Let $M$ a Markov source and $n$ an integer.
The Markov Independent Model (MI) considers $n$ infinite 
words generated by $M$. The choice of the starting symbol of each sequence
is a parameter of the model. For example, all the sequences
might start with the same letter of the alphabet $c \in \mathcal{A}$.
Or the first symbol could be initialized using the stationary distribution
of the Markov chain related to $M$.

\vspace{\baselineskip}
This is an example for $n = 4$. These are the sequences :

\[
\begin{array}{cl}
  X(1) &= 00000\dots \\
  X(2) &= 1010101\dots \\
  X(3) &= 1001101\dots \\
  X(4) &= 001100111\dots
\end{array}
\]

These sequences are used to build a digital search tree
by considering the shortest prefix of each sequence 
that has not appeared yet in the previously considered sequences.

On our example, it yields the parsed word :
  \centers{$()(0)(1)(10)(00)$}
which can be read as the DST :
  \centers{ INSERT TREE }

\section{Tail symbols}

\subsection{Definition}

Each of the $n$ sequences possess a tail
symbol. For each sequence, its tail symbol is the character that
immediately follows the prefix phrase inserted into the DST.
Therefore the tail symbol is a specific character of this sequence.
If we only have the DST containing the prefix phrases, we cannot
recover the tail symbols.

Visually, with the \textcolor{red}{prefix phrases} in red and the 
\textcolor{green}{tail symbol} in green :

\begin{egalites}
  & X(1) 
    & {\color{red}{0}} {\color{green}{0}} 00000\dots \\
  &X(2) 
    & {\color{red}{1}} {\color{green}{0}}10101\dots \\
  & X(3) 
    & {\color{red}{10}} {\color{green}{0}} 1101 \dots \\
  & X(4) 
    & {\color{red}{00}} {\color{green}{1}} 100111\dots
\end{egalites}


Let $c$ be a character from our alphabet $\{ a, b \}$.
In the case when all the sequences start with a $c$, we 
define ${T_n}^{c}$ the \emph{number of times $a$ is a tail symbol in 
the experiment}.

\subsection{Recurrence relation}

For $n \geq 0$, we have :
    \[ \boxed{ {T_{n+1}}^{c} = \delta_a + 
                            {{\tilde T}_{N_a}}^a
                            + {{\tilde T}_{N_b}}^b } \]

where :
\begin{itemize}
  \item $\delta_a = 
            \begin{cases} 
                1 & \text{if $a$ is the tail symbol of the
                          first sequence}\\
                0 & \text{else} 
              \end{cases}$

  \item $N_a$ is the 
random variable giving \emph{the size of the left subtree 
which contains phrases whose second letter is $a$}

  \item ${{\tilde T}_{N_a}}^a$ is the number of 
times $a$ is a tail symbol for the sequences that were 
used to  build the subtree with phrases having $a$ as second
symbol.

  \item $T_0^c$ for all $c$ by convention.
\end{itemize}

If we take
$\{ N_a = k \}$, then $\{ N_b = n - k \}$, then
the count of the tail symbols on the left tree 
is independent of the one on the right tree :
\textit{i.e.} these quantities are conditionnaly independent.

\section{ Total path length }

\subsection{ Definition }

Defining ${L_n}^c$ as the \emph{total 
path length of the nodes of the DST that was built with
MI model with $n$ sequences starting with letter $c$}.
It is the sum of the lengths of all the prefix phrases.

\subsection{ Recurrence relation }

There is another recursive stochastic relation for 
this quantity, which is, for all $n\geq 0$ :

\[
  \boxed{ 
    {L_{n+1}}^c = n + 
                        {{\tilde L}_{N_a}}^a + 
                        {{\tilde L}_{N_b}}^b
  }
  \]

with the convention that $L_0^c = 0$ for all $c$.

Same as for the number of tail symbols, this relation 
is found by considering the DST and its two main
subtrees. Except that this time, we count the number 
of times an edge contributes to the path length.
The root with its two nodes contributes as $n$.
The two subtrees contribute respectively for 
${{\tilde L}_{N_a}}^a$ and ${{\tilde L}_{N_b}}^b$.

It is convenient that these two quantities are conditionnaly
independent in the same way as previously seen for the tail symbols.


\section{ Poisson tranform differential equation }

% We want to estimate $\Cov({T_n}^c, {L_n}^c)$.
\subsection{ Covariance recurrence relation }

Using the previous recurrence relations, we have for all $n\geq 0$ :

\centers{$ \Cov({T_{n+1}}^c, {L_{n+1}}^c )
           = \Cov( \delta_a + 
                            {{\tilde T}_{N_a}}^a
                            + {{\tilde T}_{N_b}}^b 
                   , n + 
                        {{\tilde L}_{N_a}}^a + 
                        {{\tilde L}_{N_b}}^b) $}

Since the covariance is a bilinear function which is equal
to zero if its two terms are independent or if one is constant,
we can ignore the term $n$ and expand this quantity into six terms :

\vspace{\baselineskip}
$
\begin{array}{rl}
   \Cov({T_{n+1}}^c, {L_{n+1}}^c) 
    &
            = \Cov( \delta_a, {{\tilde L}_{N_a}}^a )
              + \Cov (\delta_a,  {{\tilde L}_{N_b}}^b)
              + \Cov ( {{\tilde T}_{N_a}}^a,
                         {{\tilde L}_{N_a}}^a ) \\[2mm]
    & \,\,
              + \Cov ( {{\tilde T}_{N_a}}^a, 
                          {{\tilde L}_{N_b}}^b )
              + \Cov ( {{\tilde T}_{N_b}}^b,
                          {{\tilde L}_{N_a}}^a ) 
              + \Cov ( {{\tilde T}_{N_b}}^b, 
                          {{\tilde L}_{N_b}}^b )
\end{array}
$
\vspace{\baselineskip}


Since $\delta_a$ is given by the tail symbol of the first sequence, which
is independent from the rest of the process :
$  \Cov( \delta_a, {{\tilde L}_{N_a}}^a ) = 
       \Cov( \delta_a, {{\tilde L}_{N_b}}^b ) = 0 $

\pagebreak
Now, it is not obvious if the pairs $({{\tilde T}_{N_a}}^a, {{\tilde L}_{N_b}}^b)$
and $({{\tilde T}_{N_b}}^b, {{\tilde L}_{N_a}}^a)$ are independent or 
uncorrelated, because the 
random variable $N_a$ is not fixed. However they are conditionnaly independent,
therefore :

\vspace{\baselineskip}

$
\begin{array}{cl}
   \Cov ( {{\tilde T}_{N_a}}^a, 
                          {{\tilde L}_{N_b}}^b )
      & = \Sum{k=0}{n} \Cov ( {{\tilde T}_{N_a}}^a, 
                          {{\tilde L}_{N_b}}^b  \, | \, N_a = k) P(N_a = k) \\[2mm]
      & = \Sum{k=0}{n} \Cov ( {{\tilde T}_{k}}^a, 
                          {{\tilde L}_{n-k}}^b ) P(N_a = k) \\[2mm]
      & = \Sum{k=0}{n} 0 \cdot P(N_a = k) \\[2mm]
      & = 0
\end{array}
$
\vspace{\baselineskip}

Samely, $\Cov ( {{\tilde T}_{N_b}}^b,
                          {{\tilde L}_{N_a}}^a ) = 0$.
Yielding :

  \[ \boxed{ \Cov({T_{n+1}}^c, {L_{n+1}}^c) = 
          \Cov ( {{\tilde T}_{N_a}}^a,
                         {{\tilde L}_{N_a}}^a )
          + \Cov ( {{\tilde T}_{N_b}}^b, 
                          {{\tilde L}_{N_b}}^b ) } \]


\subsection{ Poisson transform }

Defining 
  \[ \boxed{ C_c(z) 
            = \Sum{n\geq0}{} \Cov({T_n}^c, {L_n}^c) 
                            \f{z^n}{n!} \ex{-z} } \]

Computing, with $p = P(a | c)$ and $q = 1-p$ :

\[
\begin{array}{cl}
  \SUM{n\geq 0}{} \Cov ( {{\tilde T}_{N_a}}^a, 
                          {{\tilde L}_{N_a}}^a ) \f{z^n}{n!} \ex{-z} 
      & = \SUM{n\geq 0}{} \Sum{k=0}{n} P(N_a = k) \Cov ( {{\tilde T}_{N_a}}^a, 
                          {{\tilde L}_{N_a}}^a \, | \, N_a = k) \f{z^n}{n!} \ex{-z} \\
      & = \SUM{n\geq 0}{} \Sum{k=0}{n} \binom{n}{k} p^k q^{n-k} 
                          \Cov ( {{\tilde T}_k}^a, 
                          {{\tilde L}_{k}}^a) \f{z^n}{n!} \ex{-z} \\
\end{array}
\]


In this case, $ {{\tilde T}_k}^a $ and $ T_k^a $
as well as $  {{\tilde L}_{k}}^a $
and $  L_k^a$ 
have the same distribution, hence :

\centers{$
\begin{array}{rl}
  \SUM{n\geq 0}{} \Cov ( {{\tilde T}_{N_a}}^a, 
                          {{\tilde L}_{N_a}}^a ) \f{z^n}{n!} \ex{-z} 
      & = \SUM{n\geq 0}{} \Sum{k=0}{n} \binom{n}{k} p^k q^{n-k} 
                          \Cov (  T_k^a, L_k^a)
                           \f{z^n}{n!} \ex{-z} \\
      & = \SUM{n\geq 0}{} \Sum{k=0}{n} 
              \left( \f{(zp)^k}{k!}  \Cov (  T_k^a, L_k^a ) \ex{-zp} \right)
              \left( \f{(zq)^{n-k}}{(n-k)!} \ex{-zq} \right) \\
     &= \underbrace{\left( \SUM{n\geq 0}{} \f{(zp)^n}{n!}  \Cov (  T_n^a, L_n^a ) \ex{-zp} \right)}_{= C_a(zp)}
         \underbrace{\left( \SUM{n\geq 0}{}  \f{(zq)^{n}}{n!} \ex{-zq} \right)}_{= 1} \\
      &= C_a(zp) 
\end{array}
$}

A similar computation gives $ \SUM{n\geq 0}{} \Cov ( {{\tilde T}_{N_b}}^b, 
                          {{\tilde L}_{N_b}}^b ) \f{z^n}{n!} \ex{-z} = C_b(zq) $, 
this time conditionning on $P(N_b = k) = \binom{n}{k} q^k p^{n-k} $.

From what we've seen, when derivating $C_c(z)$ we get :

\[
\begin{array}{cl}
\partial_z C_c(z) 
      &= \SUM{n\geq0}{} \Cov(T_n^c, L_n^c) n \f{z^{n-1}}{n!} \ex{-z} 
         - C_c(z) \\
      &= \SUM{n\geq0}{} \Cov({T_{n+1}}^c, {L_{n+1}}^c)  \f{z^{n}}{n!} \ex{-z} 
         - C_c(z) \\
      &= \SUM{n\geq0}{} \left[\Cov ( {{\tilde T}_{N_a}}^a,
                         {{\tilde L}_{N_a}}^a )  
          + \Cov ( {{\tilde T}_{N_b}}^b, 
                          {{\tilde L}_{N_b}}^b )\right] \f{z^n}{n!} \ex{-z} 
                - C_c(z) \\
      &= C_a(zp) + C_b(zq) - C_c(z)
\end{array}
\]

Finally the equation for $C_c(z)$ is :

\[
  \boxed{
  \partial_z C_c(z) + C_c(z) 
      = C_a(zp) + C_b(zq)
  }
\]


% Pas de \end{document} ici: voir 'main.tex'
