
\centers{\question{Alternative representation}}

\noindent
Another expression for the variance might be

\centers{$  \left( \ddot{\lambda}(-1) - { \dot{\lambda}(-1) }^2 \right) \f{n}{\log_2^2 n} $}

\noindent I was able to compute this coefficient for a Markov chain of size 2. It 
might be applied to the general case~$n$ by solving a linear system of the same size
and computing an approximation of the largest eigenvalue $\lambda$. 

\leftcenters
    {In the paper,}
    {$ \ddot{\lambda}(-1) = \pi \ddot{P}(-1)\psi
                        + 2 \dot{\pi}(-1) \dot{P}(-1) \psi
                        - 2 \dot{\lambda}(-1) \dot{\pi}(-1) \psi $}
\noindent However, the relations defining $\pi(s)$ : 
   
        \centers{ $ \left\{
            \begin{array}{rl} \pi(s) P(s)  &= \lambda(s) \pi(s) \\
                          P(s) \psi(s) &= \lambda(s) \psi(s) \\
                          \pi(s) \psi(s) &= \lambda(s) 
            \end{array}
                    \right. $}
         
did not seem to allow me to directly compute $\dot{\pi}(s)$ (it seemed like I need
one more).
Therefore, I computed $\lambda(s)$ as the greatest 
eigenvalue of $P(s)$. Let $\chi$ the characteristic polynomial,
and $\Delta$ its discrimant

\centers{$ \chi = (X - {p_{0 0}}^{-s}) 
                  (X - {p_{1 1}}^{-s}) 
                    - (p_{0 1} \, p_{1 0})^{-s} $}

\centers
    {$ \Delta = (\poo + \pii)^2 - 4[\pooii - \poiio] = {p_{0 0}}^{-2s} 
                + {p_{1 1}}^{-s} - 2\pooii + 4\poiio $}

\noindent We need to compute $\delta$, a complex root of 
$\Delta$, in order to find the roots of $\chi$. First, we
write $\Delta$ as :

\begin{calculs}
    & \Delta 
        &=& \underbrace{p_{0 0}^{-2\Re(s)} \cos(2\ln(p_{0 0})\Im(s))}_{a_0(s)} \\[6mm]
          &&+& \underbrace{p_{1 1}^{-2\Re(s)} \cos(2\ln(p_{1 1})\Im(s))}_{a_1(s)} \\[6mm]
           &&& \underbrace{- 2 (p_{0 0}\,p_{1 1})^{-\Re(s)} \cos(\ln(p_{0 0}\,p_{1 1})\Im(s))}_{a_2(s)} \\[6mm]
           &&+&  \underbrace{4 (p_{0 1}\,p_{1 0})^{-\Re(s)} \cos(\ln(p_{0 1}\,p_{1 0})\Im(s))}_{a_3(s)} \\[6mm]
          &&+& i \Im(\Delta) 
\end{calculs}

where $\Im(\Delta) = b_0(s) + b_1(s) + b_2(s) + b_3(s)$, with each $b_i(s)$ being
the same term as $a_i(s)$ with $\cos$ replaced by $\sin$. We write 

\centers{$\Delta = \alpha(s) + i \beta(s)$}

and we are searching for $\delta = x(s) + i y(s)$, which means that 

\centers{$ \left\{
                \begin{array}{rl}
                    x^2 - y^2 &= \alpha \\
                    2\, x\, y &= \beta \\
                    x^2 + y^2 &= \sqrt{\alpha^2+\beta^2}
                \end{array}
            \right.
        $}
    

Yielding
 
    \centers{$
        \left\{
            \begin{array}{rl}
                     x &= \pm \sqrt{\f{1}{2}(\sqrt{\alpha^2+\beta^2}+\alpha)} \\
                     y &= \pm \sqrt{\f{1}{2}(\sqrt{\alpha^2+\beta^2}-\alpha)} 
            \end{array}
        \right. 
            $}

And since $2xy = \beta$, we have $\epsilon \in \{-1,1\}$ such that
             \centers{$ \delta = \pm (x + \ir \epsilon y) $}

\leftcenters    
    {So}
    {$\lambda(s) = \f{ \poo + \pii \pm (x + \ir \epsilon y)}{2}$}

\leftcenters
    {\text{i.e.}}
    {$\ddot{{\lambda}}(-1) = \f{ p_{0 0} \ln^2(p_{0 0})
                                    + p_{1 1} \ln^2(p_{1 1})
                                    \pm (\ddot{{x}}(-1) + \ir \epsilon \ddot{{y}}(-1))}
                                  {2} $}
where we'll have to find what is $\epsilon$ and which sign to pick.

But first, computing the derivatives of $x(s) = \sqrt{f(s)} $:
    \centers{$\dot{x}(s) = \f{f'(s)}{2x(s)}$}
    \leftcenters
        {and}
        {$\ddot{{x}}(s) = \f{ f''(s) x(s) - f'(s) \cdot \f{f'(s)}{2x(s)} }
                                   { 2 x^2(s) } $}

Computing $f(s)$ :

    \centers{$ f(s) = \f12 (\sqrt{\alpha^2+\beta^2} + \alpha) $}
    \centers{$ f'(s) = \f12 \left[ \f{ \overbrace{ \dot{\alpha}\alpha + \dot{\beta}\beta }^{ \gamma(s) } }
                                     { \underbrace{\sqrt{\alpha^2 + \beta^2}}_{\kappa(s)} }
                                    + \dot{\alpha} \right] $}

    \leftcenters{with}{$ \dot{\alpha} = \dot{a_0} + \dot{a_1} + \dot{a_2} + \dot{a_3} $}

\leftcenters
    {As for $f''(s) :$}
    {$ f''(s) = \f12 \left[ 
                        \f{ \dot{\gamma}(s) \kappa(s) - \gamma(s) \dot{\kappa}(s) }
                          {\kappa^2(s)} 
                        + \ddot{{\alpha}}(s) 
                    \right] $}

\leftcenters
    {with}
    {$\dot{\gamma}(s) = \ddot{{\alpha}} \alpha + {\dot{\alpha}}^2 + \ddot{{\beta}}\beta + {\dot{\beta}}^2$}

\centers
    {$ \dot{\kappa}(s) = \f{2 \alpha \dot{\alpha} + 2 \beta \dot{\beta}}
                           {2\sqrt{\alpha^2+\beta^2}}$}


Deriving according to $s$ amounts to deriving according to $\Re(s)$ so, in $s=-1$,

    \centers{$ \dot{\alpha}(-1) = -2\ln p_{0 0} a_0(-1)
                              -2\ln p_{1 1} a_1(-1)
                              -\ln q_0 a_2(-1)
                              -\ln q_1 a_3(-1) $}
\leftcenters
    {and}
    {$ \ddot{\alpha}(-1) = 4\ln^2 p_{0 0} a_0(-1)
                             + 4\ln^2 p_{1 1} a_1(-1)
                              +\ln^2 q_0 a_2(-1)
                              +\ln^2 q_1 a_3(-1) $}

At this point we have fully determined $\ddot{{x}}(s)$, and we realize two things:

\begin{enumerate}
    \item In $s=-1$, since $Im(-1) =0$ and because of the sinus function,
          all the $\beta$ terms, including derivatives, are equal to 0.
          This will simplify the expression for $\ddot{{x}}(-1)$. \\

    \item Furthermore, it also means that $\ddot{{y}}(-1) = 0$,
          so
          \encadre{ $\ddot{{\lambda}}(-1) = \f{ p_{0 0} \ln^2(p_{0 0})
                                    + p_{1 1} \ln^2(p_{1 1})
                                    + \ddot{{x}}(-1) }
                                  {2} $ }
          where the $+$ comes from the fact that $\lambda(s)$ is 
          the highest eigenvalue (and $\ddot{{x}}(-1) > 0$, so by
          continuity the expression around $s=-1$ retained the same sign)
\end{enumerate}

The final expression of $\ddot{{\lambda}}$ (as well as $\dot{\lambda}(-1))$)
can be fully expressed with $\alpha(-1), \dot{\alpha}(-1)$ and $\ddot{{\alpha}}(-1)$.
I empirically verified that $\dot{\lambda}(-1) = h$, which is encouraging. It could 
mean that if there is a mistake in my computations, it happens in the computation 
of either $\ddot{x}(s)$, $f''(s)$ or $\ddot{\alpha}$.
This is the code that computes 
\centers{$\f{\ddot{\lambda}(-1)-{\dot{\lambda}(-1)}^2}{{\dot{\lambda}(-1)}^3}$}
where \verb|der_lamb| reads as $\dot{\lambda}(-1)$, etc.

\vspace{1em}
{\centering
\begin{lstlisting}[language=Python]
def compute_lambda(M):
    p00 = M[0, 0]
    p01 = M[0, 1]
    p11 = M[1, 1]
    p10 = M[1, 0]

    q0 = p00 * p11
    q1 = p01 * p10

    alpha = p00 ** 2 + p11 ** 2 - 2 * q0 + 4 * q1

    der_alpha = - 2 * log(p00) * (p00 ** 2) - 2 * log(p11) * (p11 ** 2) \
                + 2 * q0 * log(q0) \
                - 4 * q1 * log(q1)

    der2_alpha = 4 * (log(p00) ** 2) * (p00 ** 2) \
                 + 4 * (log(p11) ** 2) * (p11 ** 2) \
                 - 2 * (log(q0) ** 2) * q0 \
                 + 4 * (log(q1) ** 2) * q1

    beta = 0
    der_beta = 0
    der2_beta = 0

    gamma = der_alpha * alpha # beta terms are zero
    der_gamma = der2_alpha * alpha + (der_alpha ** 2) # beta terms are zero

    kappa = sqrt(alpha**2 + beta**2) # basically alpha
    der_kappa = (alpha * der_alpha) / (sqrt(alpha**2))

    f = 0.5 * (sqrt(alpha ** 2 + beta ** 2) + alpha)
    der_f = 0.5 * ((der_alpha * alpha) / (sqrt(alpha**2)) + der_alpha)
    der2_f = 0.5 * ( (der_gamma * kappa + gamma * der_kappa) / (kappa ** 2) + der2_alpha )

    x = sqrt(alpha)
    der_x = der_f / (2 * x)
    der2_x = (der2_f * x + der_f * der_x) / (2 * (x**2))

    lamb = 0.5 * (p00 + p11 + x)
    der_lamb = 0.5 * (- log(p00) * p00 - log(p11) * p11 + der_x)
    der2_lamb = 0.5 * ( (log(p00)**2) * p00 + (log(p11)**2) * p11 + der2_x )

    # Verifying der_lamb = h (entropy)
    h = 0
    o = M[1, 0] + M[0, 1]
    p = [M[1, 0] / o, M[0, 1] / o]
    n = len(M)

    for i in range(n):
        for j in range(n):

            h -= p[i] * M[i, j] * log(M[i, j])

    assert(der_lamb == h) # fails if der_lamb is not equal to h

    var_coeff = (der2_lamb - der_lamb ** 2) / (der_lamb**3)
    
    return var_coeff
\end{lstlisting}
}
 
% \noindent First term is
% \centers
%     { $\pi_0 \, p_{0 0} \, \log^2 (p_{0 0}) 
%         + \pi_1 \, p_{1 0} \, \log^2 (p_{1 0}) 
%         + \pi_0 \, p_{0 1} \, \log^2 (p_{0 1}) 
%         + \pi_1 \, p_{1 1} \, \log^2 (p_{1 1})  $}

% \noindent The second is
% \centers
%     { $ - 2 \pac{
%             \dot{\pi}_0(-1) p_{0 0} \log (p_{0 0})    
%             + \dot{\pi}_1(-1) p_{1 0} \log (p_{1 0}) 
%             + \dot{\pi}_0(-1) p_{0 1} \log (p_{0 1})
%             + \dot{\pi}_1(-1) p_{1 1} \log (p_{1 1})
%         }   
%     $}

% \noindent The third is
% \centers{
%     $ - 2 \dot{\lambda}(-1) \pac{
%         \dot{\pi}_0(-1) 
%         + \dot{\pi}_1(-1)
%     }$
% }

% \noindent We 
%         have to compute $\dot{\pi}(-1)$. With 
%             $\pi(s) = (\pi_0(s), \, \pi_1(s))$, and since

% \centers{$\pi(s) P(s) = \lambda(s) \pi(s)$}

% \leftcenters
%     {then we have}
%     {$ \left\{
%         \begin{aligned}
%             {p_{0 0}}^{-s} \pi_0(s) + {p_{0 1}}^{-s} \pi_1(s) &= \lambda(s) \pi_0(s) \\
%             {p_{1 0}}^{-s} \pi_0(s) + {p_{1 1}}^{-s} \pi_1(s) &= \lambda(s) \pi_1(s) \\
%         \end{aligned}
%         \right.
%      $}

% which I'm not sure how to solve formally.
% \noindent
% solving for $\pi_0(s)$ and $\pi_1(s)$, 
% assuming $ p_{0 0} p_{1 1} \neq p_{0 1} p_{1 0} $ :

% \centers{$ \pi_0(s) = \lambda(s) \f{ \overbrace{{p_{1 1}}^{-s} - {p_{0 1}}^{-s}}^{g_0(s)} }
%                                    { \underbrace{{ (p_{0 0} p_{1 1}) }^{-s} 
%                                         - { (p_{0 1} p_{1 0}) }^{-s}}_{\delta(s)} }
%             \qquad 
%             \text{and}
%             \qquad
%        \pi_1(s) = \lambda(s) \f{ \overbrace{{p_{0 0}}^{-s} - {p_{1 0}}^{-s}}^{g_1(s)} }
%                                    { { (p_{0 0} p_{1 1}) }^{-s} 
%                                         - { (p_{0 1} p_{1 0}) }^{-s} }$}


% \leftcenters
%     {hence for $i\in\{0,1\}$}
%     {$ \dot{\pi}_i(s) = \dot{\lambda}(s) \f{g_i(s)}{\delta(s)}
%                         + \lambda(s) \f{{g_i}'(s)\delta(s) - g_i(s)\delta'(s)}
%                                         {\delta^2(s)}$}

% \leftcenters
%     {so}
%     {$ \dot{\pi}_i(-1) = h \f{g_i(-1)}{\delta(-1)}
%                         + \f{{g_i}'(-1) \delta(-1) - g_i(-1) \delta'(-1) }
%                             {\delta^2(-1)} $}

% \begin{egalites}
%     with & \delta(-1) 
%             & p_{0 0} p_{1 1} - p_{0 1} p_{1 0} \\[3mm]
%          &g_0(-1) 
%             & p_{1 1} - p_{0 1} \\[3mm]
%         &{g_0}'(-1)
%             & -\ln (p_{1 1}) p_{1 1} + \ln (p_{0 1}) p_{0 1} \\[3mm]
%         &g_1(-1)
%             & p_{0 0} - p_{1 0} \\[3mm]
%         &{g_1}'(-1) 
%             & -\ln (p_{0 0}) p_{0 0} + \ln (p_{1 0}) p_{1 0} 
% \end{egalites}


% \leftcenters
%     {where}
%     {  $\left\{
%         \begin{minipage}{0.6\textwidth}
%        $ {g_0}'(s) = - \ln(p_{1 1}) {p_{1 1}}^{-s}
%                  + \ln(p_{0 1}) {p_{0 1}}^{-s}$ \\[3mm]
%         $\delta'(s) = - \ln( p_{0 0} p_{1 1}) { (p_{0 0} p_{1 1}) }^{-s}
%                         +  \ln ( p_{0 1} p_{1 0} ) { (p_{0 1} p_{1 0}) }^{-s} $
%         \end{minipage}
%         \right.$ }



% \noindent
% Now taking $s=-1$ and re-arranging in a linear system of unknown $\dot{\pi}_0(-1)$ and $\dot{\pi}_1(-1)$ :
% \centers
%     {$ \left\{
%         \begin{aligned}
%             \dot{\pi}_0(-1) (p_{0 0} - \lambda)
%             + \dot{\pi}_1(-1) p_{0 1}
%                 &= \dot{\lambda}(-1) \pi_0
%                     +   \ln p_{0 0} \cdot p_{0 0} \pi_0
%                     +   \ln p_{0 1} \cdot p_{0 1} \pi_1 \\
%             \dot{\pi}_0(-1) p_{1 0}
%             + \dot{\pi}_1(-1) (p_{1 1} - \lambda)
%                 &= \dot{\lambda}(-1) \pi_1
%                     +  \ln p_{1 0} \cdot p_{1 0} \pi_0 
%                     +  \ln p_{1 1} \cdot p_{1 1} \pi_1
%         \end{aligned}
%         \right.
%      $}

% \noindent since
% $\dot{\lambda}(-1) = h$ and,
% $\lambda(-1) = 1$ :

% \centers
%     {$ \left\{
%         \begin{aligned}
%             \dot{\pi}_0(-1) (p_{0 0} - 1)
%             + \dot{\pi}_1(-1) p_{0 1}
%                 &= h \pi_0
%                     +   \ln p_{0 0} \cdot p_{0 0} \pi_0
%                     +   \ln p_{0 1} \cdot p_{0 1} \pi_1 \\
%             \dot{\pi}_0(-1) p_{1 0}
%             + \dot{\pi}_1(-1) (p_{1 1} - 1)
%                 &= h \pi_1
%                     +  \ln p_{1 0} \cdot p_{1 0} \pi_0 
%                     +  \ln p_{1 1} \cdot p_{1 1} \pi_1
%         \end{aligned}
%         \right.
%      $}

% \begin{egalites}
% which is  &\lambda
%         &\f{ (p_{0 0} + p_{1 1}) 
%            + \sqrt{ (p_{0 0} + p_{1 1})^2 - 4 (p_{0 0} p_{1 1} - p_{1 0} p_{0 1}) }}
%            {2}\\[3mm]
%         &&\f{ (p_{0 0} + p_{1 1}) 
%            + \sqrt{ (p_{0 0} - p_{1 1})^2 + 4 p_{1 0} p_{0 1} }}
%            {2}   
% \end{egalites}





