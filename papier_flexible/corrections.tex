\question{Corrections}

\begin{itemize}

\item The formal definition of ${D_n}^{\text{LZ}}$ in $\numero{3}$ is 
      misleading because it contradicts the previous verbal definition.
      This formula is the empirical average length of a phrase, whereas
      ${D_n}^{\text{LZ}}$ is used in the rest of the paper as the length
      of a randomly selected phrase. These two aren't equal :
      if we build a Lempel-Ziv DST from a word, then 
      ${D_n}^{\text{LZ}}$ can be seen as the depth of a random node, 
      which is different from the average path length computed 
      on all the nodes. 

\item In \emph{Remark 2}, I believe the definition of $v$ and 
      $t$ should rather be 
      $ v = a_{i'} \dots a_{i} $ and
      $ t = a_{i+1} \dots a_n$. 

\item The result $\numero{14}$ should be an equality, and actually it is one
      because of the flexible parsing algorithm. A proof by contradiction
      can show this. However, we upper bound 'g(j)' by $\pinf$ in 
      $\numero{23}$ so I currently see no purpose to using $\numero{14}$ (as an 
      equality) instead of $\numero{13}$. Furthermore, an equality is 
      needed to establish result $\numero{22}$.

\item The proof and reasoning around \emph{Theorem 1} has several flaws.
      The notation $X$ for a sequence depending on $n$, which isn't a 
      random variable, is misleading. On the other hand, it should appear
      that both $g$ and $j$ are random variables, as the randomness of 
      $j$ is used in the end of the proof. As for the arguments that 
      link $|L_{g(j)-k}|$ to ${D_n}^{\text{LZ}}$, I have
      indicated how I think they should be developped in the next section.

\item \emph{Theorem 2} is false as stated: we proved \emph{Theorem 1}
      using a random $j$. The randomness remains, so the quantifier
      'for any $j < M_n$' should be removed. This would be true for 
      \emph{Theorem 1} too.

\item The proof of \emph{Theorem 2} may stop at $\numero{26}$
      since we can directly prove this upperbound goes to $0$.
      This yields a tighter upperbound for \emph{Theorem 2}.
      I detailed this analysis in the last section of this report.

\item In that same proof, the step between $\numero{25}$ and 
      $\numero{26}$ relies heavily on a result from $\pac{6}$.
      It seems not obvious that the result can be applied as such.

\end{itemize}