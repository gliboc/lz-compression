\titre{Optimal parsing of LZ78 for memoryless sources}


\separation
\bk


\question{Notations}
Let $W$ be the random variable which outputs words of length $n$ ($\Omega_n$)
from a memoryless source. Let $J \in \mathbb{N}^{\star}$ be a random
variable which, for every word $w$, uniformly picks the index of
one of its phrases. The joint law of $J$ with $W$ is :

\centers{$ \proba{ W=w, J=j } = 
                    \left\{ \begin{array}{l}
                        \tf{1}{M_n(w)} \quad \textmd{ if } j \leq M_n(w) \\
                        0 \qquad \qquad \textmd{ else }
                    \end{array}
                    \right. $ }


\noindent
The function defined as $g$ is a random variable that depends on $W$, therefore
we write it as $g_W$, or, during the event $\{ W=w \}$, $g_w$. Samely, we will
study the random variable $g_W(J)$, and will
sometimes write $g_w(j)$ when both $W$ and $J$ are determined as $w$ and $j$.


\leftcenters
    {We define the random variable}
    {$B_{J, W}^k = | L_{g_W(J) - k} |$}

\noindent
and we will study the random variable
    {$ \underset{ 0 \leq k \leq g_W(J) }{ \max } 
                    \left\{ { B_{J, W}^k - k } \right\} $}

\noindent
For $J = j$, $W = w$ and $j \leq M_n(w)$, this random variable is the length of 
the $j^{\textmd{th}}$ phrase of the flexible parsing of the word $w$. We have,
accordingly,

\centers{$ D_n^{FLEX} (w) = \f{1}{M_n(w)} \Sum{j=0}{M_n(w)-1} |u_j^{FLEX}|
                          = \f{1}{M_n(w)} \Sum{j=0}{M_n(w)-1}  
                                \underset{ 0 \leq k \leq g_w(j) }{ \max } 
                                \left\{ { B_{j, w}^k - k } \right\} $}
\noindent
Precisely, $D_n^{FLEX}(w)$ is the empirical average of the lengths of the phrases
of the flexible parsing of a word $w$, whereas 
        $\underset{ 0 \leq k \leq g_w(J) }{ \max } 
        \left\{ { B_{J, w}^k - k } \right\} $,
with $J$ a random variable and $w$ fixed,
is the length of a uniformly randomly selected phrase of the flexible parsing
of $w$. 

\leftcenters   
    {Finally, we define}
    {$x_n = \f1{h} \log_2 \pa{ \f{nh}{\log_2(n)} }$}

\leftcenters
    {as well as}
    {$b_n^{\delta} = x_n + (c_3 x_n)^{\delta}$}

\leftcenters
    {and we study}
    {$ \proba{ 
        \underset{ 0 \leq k \leq g_w(J) }{ \max } 
        \left\{ { B_{J, w}^k - k } \right\}
            > b_n^{\delta} } $}


\question{Computations}
\noindent By conditionning on $W$ and $J$, we obtain

\begin{calculs}
        & \proba{ B_{J, W} > b_n^{\delta} }
            &=& \SUM{w \in \Omega_n}{} \SUM{j \in \mathbb{N}^{\star}}{} \,
                \proba{ B_{J, W} > b_n^{\delta} \,|\, W=w, J=j }
                \proba{ W=w, J=j } \\
            
            &&=& \SUM{w \in \Omega_n}{} \SUM{j \in \mathbb{N}^{\star}}{} \,
                \proba{ \Union{k=0}{g_w(j)} \left\{ B_{w, j}^k > k + b_n^{\delta} \right\} 
                            \,|\, W=w, J=j }
                \proba{ W=w, J=j } \\

            &&\leq&
                \SUM{w \in \Omega_n}{} \SUM{j \in \mathbb{N}^{\star}}{}
                \SUM{k=0}{g_w(j)} 
                \proba{ B_{w, j}^k > k + b_n^{\delta} \,|\, W=w, J=j }
                \proba{ W=w, J=j } \\

            &&\leq&
                \SUM{w \in \Omega_n}{} \SUM{j \in \mathbb{N}^{\star}}{}
                \SUM{k=0}{\pinf} \,
                \proba{ B_{w, j}^k > k + b_n^{\delta} \,|\, W=w, J=j }
                \proba{ W=w, J=j } \\

            &&=&
                \SUM{k=0}{\pinf}
                \SUM{w \in \Omega_n}{} \SUM{j \in \mathbb{N}^{\star}}{} \,
                \proba{ B_{w, j}^k > k + b_n^{\delta} \,|\, W=w, J=j }
                \proba{ W=w, J=j } \\

            &&=&
                \SUM{k=0}{\pinf} \,
                \proba{ B_{W, J}^k > k + b_n^{\delta} } \\
\end{calculs}




\noindent
For all $k\in\mathbb{N}$, we are now studying the probability :
    \centers{
        $ \proba{ B_{J, W}^k > k + b_n^{\delta} } $
    }

\noindent
The first








\question{Proving the upper bound goes to 0}
Assuming we have

\centers{ $ \Sum{k=0}{\pinf} 
            \proba{ D_n^{LZ} (W) > k + b_n^{\,\delta} }
            \leq 
            A \Sum{i=0}{\pinf} \alpha^{ \f{ i }{ \sqrt{ c_3 x_n } } }
        $}

\noindent Since $\alpha < 1$, the sum of the terms of a geometric sequence gives

\centers{ $ \Sum {i=0}{\pinf}
            \alpha^{ \f{i}{c_3 x_n} } =
                  \f { 1 } 
                     { 1 - \alpha^{ \f 1 { \sqrt{ c_3 x_n } } } } $}

\noindent Here, it would weaken the result to give an upper bound since
we just want to prove that this bound goes to 0 for $n\rightarrow\pinf$.
For that, we prove that
    \centers{$ \underset{n\rightarrow\pinf}{\limt} \,\,\,
                  \f { \alpha^{ {(c_3 x_n)}^{ \delta - \tf12} } } 
                     { 1 - \alpha^{ \f 1 { \sqrt { c_3 x_n } } } = 0 }
            $}

\leftcenters
    {Let's define}
    {$ f(x) = \alpha^{ x^ { \delta - \tf12 } } $}

\leftcenters
    {and}
    {$ g(x) = 1 - \alpha^{ \f 1 { \sqrt{x} } } $}

\noindent Their derivates are

\centers{$ f'(x) = \ln \alpha \pa {\delta - \f12}
                    \, x^{\delta - \tf32}
                    \, f(x) $}

\leftcenters{and}
    {$ g'(x) 
        =  \ln \alpha\,\, \f 1 {2x\sqrt{x}}\, \alpha ^ {\f 1 {\sqrt{x}}} $}

\noindent Therefore

   \centers {
        $ \f{f'(x)}{g'(x)}
             = \f { \pa{\delta - \f12}
                    \, x^{\delta - \tf32}
                    \alpha^{ x^{ \delta - \tf12 } } }
                  { \f{1}{2x\sqrt{x}} 
                        \alpha^{ \f{1}{\sqrt{x}} } }
        = 2 \pa{\delta -\f12} x^{\delta} 
            \times \f { \alpha^{x^{\delta-\tf12}}}
                      { \alpha^{ \f{1}{\sqrt{x}} } } $}  
                    
