

\question{Computations}
\noindent By conditionning on $W$ and $J$, we obtain

\begin{calculs}
        & \proba{ B_{J, W} > {b_n}^{\delta} }
            &=& \SUM{w \in \Omega_n}{} \SUM{j \in \mathbb{N}^{\star}}{} \,
                \proba{ B_{J, W} > {b_n}^{\delta} \,|\, W=w, J=j }
                \proba{ W=w, J=j } \\
            
            &&=& \SUM{w \in \Omega_n}{} \SUM{j \in \mathbb{N}^{\star}}{} \,
                \proba{ \Union{k=0}{g_w(j)} \left\{ B_{w, j}^k > k + {b_n}^{\delta} \right\} 
                            \,|\, W=w, J=j }
                \proba{ W=w, J=j } \\

            &&\leq&
                \SUM{w \in \Omega_n}{} \SUM{j \in \mathbb{N}^{\star}}{}
                \SUM{k=0}{g_w(j)} 
                \proba{ B_{w, j}^k > k + {b_n}^{\delta} \,|\, W=w, J=j }
                \proba{ W=w, J=j } \\

            &&\leq&
                \SUM{w \in \Omega_n}{} \SUM{j \in \mathbb{N}^{\star}}{}
                \SUM{k=0}{\pinf} \,
                \proba{ B_{w, j}^k > k + {b_n}^{\delta} \,|\, W=w, J=j }
                \proba{ W=w, J=j } \\

            &&=&
                \SUM{k=0}{\pinf}
                \SUM{w \in \Omega_n}{} \SUM{j \in \mathbb{N}^{\star}}{} \,
                \proba{ B_{w, j}^k > k + {b_n}^{\delta} \,|\, W=w, J=j }
                \proba{ W=w, J=j } \\

            &&=&
                \SUM{k=0}{\pinf} \,
                \proba{ B_{W, J}^k > k + {b_n}^{\delta} } \\
\end{calculs}




\noindent
For all $k\in\mathbb{N}$, we might prove that
    \centers{
        $ \proba{ B_{J, W}^k > k + {b_n}^{\delta} } 
            \leq \proba{ {D_n}^{\text{LZ}} > k + {b_n}^{\delta} } $
    }

\noindent
Let $k \in\mathbb{N}$.
Currently, the proof to show this stands on three arguments :

\begin{itemize}
    \item[(1)] The first is that $L_{g_W(J)-k}$ can be seen as 
          a random phrase from the Lempel-Ziv parsing of 
          a word of length $N$, where $N \leq g_W(J)$.

    \item[(2)] The second, is that $|L_{g_W(J)-k}|$ can 
          therefore be considered the same as ${D_N}^{\text{LZ}}$
          \textit{i.e} at least equal in law.

    \item[(3)] The third is that, for all $n'\leq n$, 
        ${D_n'}^{\text{LZ}} \leq {D_n}^{\text{LZ}}$.
\end{itemize}

There are different problems with each of these arguments :

\begin{itemize}
    \item The randomness of $L_{g_W(J)-k}$ is not clearly
          established. For example, what is $N$ ? In order
          to establish this result, we have to define it
          to be able to even consider ${D_N}^{\text{LZ}}$.

    \item Even if we can identify $N$ and, let's say,
          condition it with $\{ N = n' \}$, it is not obvious
          that the choice of a phrase in ${g_W(J)-k}$
          is the same as the uniform choice that operates
          in ${D_n'}^{\text{LZ}}$.

    \item As for (3), this result is true on average, but 
          not always. Indeed, since ${D_n}^{\text{LZ}}$
          (resp. ${D_n'}^{\text{LZ}}$) is concentrated
          around $x_n$ (resp. $x_n'$), and $x_n'$ < $x_n$
          since $n' < n$, we can show 
          that this result holds with high probability.
\end{itemize}








\question{Proving the upper bound goes to 0}
Assuming we have

\centers{ $ \Sum{k=0}{\pinf} 
            \proba{ D_n^{LZ} (W) > k + b_n^{\,\delta} }
            \leq 
            A \Sum{i=0}{\pinf} \alpha^{ \f{ i }{ \sqrt{ c_3 x_n } } }
        $}

\noindent Since $\alpha < 1$, the sum of the terms of a geometric sequence gives

\centers{ $ \Sum {i=0}{\pinf}
            \alpha^{ \f{i}{c_3 x_n} } =
                  \f { 1 } 
                     { 1 - \alpha^{ \f 1 { \sqrt{ c_3 x_n } } } } $}

\noindent Here, it would weaken the result to give an upper bound since
we just want to prove that this bound goes to 0 for $n\rightarrow\pinf$.
For that, we prove that
    \centers{$ \underset{n\rightarrow\pinf}{\limt} \,\,\,
                  \f { \alpha^{ {(c_3 x_n)}^{ \delta - \tf12} } } 
                     { 1 - \alpha^{ \f 1 { \sqrt { c_3 x_n } } } = 0 }
            $}

\leftcenters
    {Let's define}
    {$ f(x) = \alpha^{ x^ { \delta - \tf12 } } $}

\leftcenters
    {and}
    {$ g(x) = 1 - \alpha^{ \f 1 { \sqrt{x} } } $}

\noindent Their derivates are

\centers{$ f'(x) = \ln \alpha \pa {\delta - \f12}
                    \, x^{\delta - \tf32}
                    \, f(x) $}

\leftcenters{and}
    {$ g'(x) 
        =  \ln \alpha\,\, \f 1 {2x\sqrt{x}}\, \alpha ^ {\f 1 {\sqrt{x}}} $}

\noindent Therefore

   \centers {
        $ \f{f'(x)}{g'(x)}
             = \f { \pa{\delta - \f12}
                    \, x^{\delta - \tf32}
                    \alpha^{ x^{ \delta - \tf12 } } }
                  { \f{1}{2x\sqrt{x}} 
                        \alpha^{ \f{1}{\sqrt{x}} } }
        = 2 \pa{\delta -\f12} x^{\delta} 
            \times \f { \alpha^{x^{\delta-\tf12}}}
                      { \alpha^{ \f{1}{\sqrt{x}} } } $}  
