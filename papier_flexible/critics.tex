\noindent
\hypertarget{critics}{}
Let $k \in\mathbb{N}$.
Currently, the proof to show this stands on three arguments :

\begin{itemize}
    \item[(1)] The first is that $L_{g_W(J)-k}$ is  
          a random phrase from the Lempel-Ziv parsing of 
          a word of length $N$, where $N \leq g_W(J) \leq n$.
          In the event $\{ N = n' \}$, we consider 
          ${D_{n'}}^{\text{LZ}}$.

    \item[(2)] The second, is that $|L_{g_W(J)-k}|$ can 
          therefore be considered the same as ${D_N}^{\text{LZ}}$
          \textit{i.e} at least equal in law.

    \item[(3)] The third is that, for all $n'\leq n$, 
        ${D_n'}^{\text{LZ}} \leq {D_n}^{\text{LZ}}$.
\end{itemize}

Although they seem generally true,
there are different problems with each of these arguments :

\begin{itemize}
    \item $N$ isn't clearly established, so 
          ${D_N}^{\text{LZ}}$ isn't really known.

    \item If we can identify $N$ and, let's say,
          condition our probability with $\{ N = n' \}$, 
          it is not obvious
          that the choice of a phrase at position ${g_W(J)-k}$
          knowing $\{ N = n' \}$
          is the same as the uniform choice that operates
          in ${D_n'}^{\text{LZ}}$.

    \item As for (3), this result is true on average, but 
          not in all cases. Indeed, since ${D_n}^{\text{LZ}}$
          (resp. ${D_n'}^{\text{LZ}}$) is concentrated
          around $x_n$ (resp. $x_n'$), and $x_n'$ < $x_n$
          since $n' < n$, we can show 
          that this result holds with high probability.
\end{itemize}